%%%%%%%%%%%%%%%%%
% This is an sample CV template created using altacv.cls
% (v1.6.5, 3 Nov 2022) written by LianTze Lim (liantze@gmail.com). Compiles with pdfLaTeX, XeLaTeX and LuaLaTeX.
%
%% It may be distributed and/or modified under the
%% conditions of the LaTeX Project Public License, either version 1.3
%% of this license or (at your option) any later version.
%% The latest version of this license is in
%%    http://www.latex-project.org/lppl.txt
%% and version 1.3 or later is part of all distributions of LaTeX
%% version 2003/12/01 or later.
%%%%%%%%%%%%%%%%

%% Use the "normalphoto" option if you want a normal photo instead of cropped to a circle
% \documentclass[10pt,a4paper,normalphoto]{altacv}

%\documentclass[9pt,a4paper,ragged2e,withhyper]{altacv}
\documentclass[8pt,a4paper, withhyper]{altacv}
%% AltaCV uses the fontawesome5 and packages.
%% See http://texdoc.net/pkg/fontawesome5 for full list of symbols.

% Change the page layout if you need to
\geometry{left=1.25cm,right=1.25cm,top=1.25cm,bottom=1.25cm,columnsep=1cm}

% The paracol package lets you typeset columns of text in parallel
\usepackage{paracol}

% Allow lualatex to load font files directly
% Ref: https://www.overleaf.com/learn/latex/Questions/I_have_a_custom_font_I%27d_like_to_load_to_my_document._How_can_I_do_this%3F
\usepackage{fontspec}

% Change the font if you want to, depending on whether
% you're using pdflatex or xelatex/lualatex
\ifxetexorluatex
  % If using xelatex or lualatex:
  \setmainfont{[RobotoSlab-Regular.ttf]}
  \setsansfont{[Lato-Regular.ttf]}
  \renewcommand{\familydefault}{\sfdefault}
\else
  % If using pdflatex:
  \usepackage[rm]{roboto}
  \usepackage[defaultsans]{lato}
  % \usepackage{sourcesanspro}
  \renewcommand{\familydefault}{\sfdefault}
\fi

% Change the colours if you want to
\definecolor{SlateGrey}{HTML}{2E2E2E}
\definecolor{LightGrey}{HTML}{666666}
\definecolor{DarkPastelRed}{HTML}{450808}
\definecolor{PastelRed}{HTML}{8F0D0D}
\definecolor{GoldenEarth}{HTML}{E7D192}
\colorlet{name}{black}
\colorlet{tagline}{PastelRed}
\colorlet{heading}{DarkPastelRed}
\colorlet{headingrule}{GoldenEarth}
\colorlet{subheading}{PastelRed}
\colorlet{accent}{PastelRed}
\colorlet{emphasis}{SlateGrey}
\colorlet{body}{LightGrey}

% Change some fonts, if necessary
\renewcommand{\namefont}{\Huge\rmfamily\bfseries}
\renewcommand{\personalinfofont}{\footnotesize}
\renewcommand{\cvsectionfont}{\LARGE\rmfamily\bfseries}
\renewcommand{\cvsubsectionfont}{\large\bfseries}


% Change the bullets for itemize and rating marker
% for \cvskill if you want to
\renewcommand{\itemmarker}{{\small\textbullet}}
\renewcommand{\ratingmarker}{\faCircle}

%% Use (and optionally edit if necessary) this .tex if you
%% want to use an author-year reference style like APA(6)
%% for your publication list
% \input{pubs-authoryear.cfg}

%% Use (and optionally edit if necessary) this .tex if you
%% want an originally numerical reference style like IEEE
%% for your publication list
\input{pubs-num.cfg}

%% sample.bib contains your publications
\addbibresource{sample.bib}

\begin{document}
\name{Nham Hoai Le}
\tagline{Senior Platform Engineer}
%% You can add multiple photos on the left or right
%\photoR{2.8cm}{Globe_High}
% \photoL{2.5cm}{Yacht_High,Suitcase_High}

\personalinfo{%
  % Not all of these are required!
  \email{lehoainham@gmail.com}
  %\phone{000-00-0000}
  %\mailaddress{Åddrésş, Street, 00000 Cóuntry}
  \location{Ho Chi Minh, Vietnam}
%%   \homepage{www.nhamlh.space}
  \twitter{@nhamlh}
  \linkedin{nhamlh}
  \github{nhamlh}
  %\orcid{0000-0000-0000-0000}
  %% You can add your own arbitrary detail with
  %% \printinfo{symbol}{detail}[optional hyperlink prefix]
  % \printinfo{\faPaw}{Hey ho!}[https://example.com/]
  %% Or you can declare your own field with
  %% \NewInfoFiled{fieldname}{symbol}[optional hyperlink prefix] and use it:
  % \NewInfoField{gitlab}{\faGitlab}[https://gitlab.com/]
  % \gitlab{your_id}
  %%
  %% For services and platforms like Mastodon where there isn't a
  %% straightforward relation between the user ID/nickname and the hyperlink,
  %% you can use \printinfo directly e.g.
  % \printinfo{\faMastodon}{@username@instace}[https://instance.url/@username]
  %% But if you absolutely want to create new dedicated info fields for
  %% such platforms, then use \NewInfoField* with a star:
  % \NewInfoField*{mastodon}{\faMastodon}
  %% then you can use \mastodon, with TWO arguments where the 2nd argument is
  %% the full hyperlink.
  % \mastodon{@username@instance}{https://instance.url/@username}
}

\makecvheader
%% Depending on your tastes, you may want to make fonts of itemize environments slightly smaller
% \AtBeginEnvironment{itemize}{\small}


%% \begin{quote}
%% ``Site Reliability Engineer with more than 8+ years of experience building and delivering internal service platform for organizations.''
%% \end{quote}

%% Set the left/right column width ratio to 7:3.
\columnratio{0.7}

% Start a 2-column paracol. Both the left and right columns will automatically
% break across pages if things get too long.
\begin{paracol}{2}
\cvsection{Experience}

\cvevent{Senior DevOps Engineer}{NAB}{Dec 2022 -- now}{Vietnam}
Designed and implemented a secure and efficient Kubernetes-as-a-Service platform for NAB's internal operations on AWS EKS. Deployed to hundreds of service teams, resulting in increased productivity and streamlined processes across the organization.

\divider

\cvevent{Senior Infrastructure Engineer}{Y42}{Mar 2022 -- Ongoing}{Remote}
As a Senior Infrastructure Engineer, I successfully managed infrastructure on Google Kubernetes Engine using Terraform. I also built an observability platform using OpenTelemetry and Datadog to quickly resolve production issues. By implementing workflows using Github Actions, Skaffold, and Kustomize, I enabled service owners to confidently deploy code to production at any time, resulting in increased productivity and agility. Additionally, I developed processes to ensure compliance with SOC2 standards.
\divider

\cvevent{Senior Platform Engineer}{Employment Hero}{Oct 2018 -- Feb 2022}{Remote}
As a senior platform engineer, I designed and implemented a secure and efficient platform for Employment Hero. This included building a Kubernetes-based infrastructure on AWS for microservices, creating CI/CD pipelines using CircleCI, Flux, and Helm, and developing an observability platform with Prometheus, Opsgenie, LogDNA, and Jaeger tracing. Additionally, I built in-house tools to manage, automate, and optimize infrastructure using shell script, Ruby, and Golang. I also oversaw platform governance, including API gateway, service mesh, Postgresql, Redis, Kafka, and more.
\divider

\cvevent{DevOps Engineer}{Fossil Group}{Aug 2016 -- Sep 2018}{Vietnam}
I have successfully implemented Kubernetes on AWS using Kops and built AWS-based infrastructure for backend teams. I also migrated systems from AWS ECS to Kubernetes without downtime, resulting in increased efficiency and productivity. Through the development of in-house tools using shell script and Python, I have automated and optimized infrastructure, improving stability and efficiency. Additionally, I planned and created CI/CD processes for development teams using Jenkins and Bitbucket pipelines, further streamlining operations and increasing productivity.
\divider

\cvevent{Cloud Operation Engineer}{Posiba}{Jan 2016 - Jul 2016}{Vietnam}
As a Cloud Operation Engineer, I am managing Linux instances on AWS using Puppet and monitoring infrastructure components with Zabbix. I am adept at collecting system and application logs through the use of Fluentd, Elasticsearch, and Kibana. Additionally, I have developed in-house devops tools utilizing shell script and Python, and have assisted developers in rolling out projects using Puppet, Jenkins, and Docker. My expertise in these areas has resulted in increased efficiency and productivity within organizations, ultimately driving business outcomes and success.
\divider

\cvevent{Linux System Administrator}{FPT Telecom}{Jul 2014 -- Jan 2016}{Vietnam}
Linux administrator with expertise in network device management, DNS clusters, email gateways, and server monitoring on Linux servers. Developing Ansible playbooks for multi-server management and managing user connectivity via Squid proxy and building, managing, and backing up VMware ESXi. Utilizes Icinga2 and Cacti for network monitoring.

% use ONLY \newpage if you want to force a page break for
% ONLY the current column
%\newpage

%% Switch to the right column. This will now automatically move to the second
%% page if the content is too long.
\switchcolumn

\cvsection{Technologies used}

\cvtag{Golang}
\cvtag{Ruby}
\cvtag{Python}
\cvtag{Shell script}

\divider\smallskip

\cvtag{AWS}
\cvtag{GCP}

\divider\smallskip

\cvtag{Kubernetes}
\cvtag{Docker}
\cvtag{Helm}
\cvtag{Kustomize}
\cvtag{Flux}
\cvtag{ArgoCD}
\cvtag{Prometheus}
\cvtag{OPA}
\cvtag{Contour}
\cvtag{Istio}
\cvtag{Envoy}
\cvtag{OpenTelemetry}
\cvtag{Datadog}

\divider\smallskip

\cvtag{Microservices}
\cvtag{RESTful}
\cvtag{gRPC}
\cvtag{Postgres}
\cvtag{Kafka}

\cvsection{Languages}

\cvskill{Vietnamese}{5}
\divider

\cvskill{English}{3}

%% Yeah I didn't spend too much time making all the
%% spacing consistent... sorry. Use \smallskip, \medskip,
%% \bigskip, \vspace etc to make adjustments.
\medskip

\cvsection{Education}

\cvevent{B.Sc.\ in Computer Network}{Ho Chi Minh University of Sciences}{2010 -- 2014}{}

% \divider

%\cvsection{Referees}

% \cvref{name}{email}{mailing address}
%\cvref{Name}{Institute}{a.beta@university.edu}

\end{paracol}


\end{document}
