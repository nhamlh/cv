%%%%%%%%%%%%%%%%%
% This is an sample CV template created using altacv.cls
% (v1.6.5, 3 Nov 2022) written by LianTze Lim (liantze@gmail.com). Compiles with pdfLaTeX, XeLaTeX and LuaLaTeX.
%
%% It may be distributed and/or modified under the
%% conditions of the LaTeX Project Public License, either version 1.3
%% of this license or (at your option) any later version.
%% The latest version of this license is in
%%    http://www.latex-project.org/lppl.txt
%% and version 1.3 or later is part of all distributions of LaTeX
%% version 2003/12/01 or later.
%%%%%%%%%%%%%%%%

%% Use the "normalphoto" option if you want a normal photo instead of cropped to a circle
% \documentclass[10pt,a4paper,normalphoto]{altacv}

%\documentclass[9pt,a4paper,ragged2e,withhyper]{altacv}
\documentclass[8pt,a4paper, withhyper]{altacv}
%% AltaCV uses the fontawesome5 and packages.
%% See http://texdoc.net/pkg/fontawesome5 for full list of symbols.

% Change the page layout if you need to
\geometry{left=1.25cm,right=1.25cm,top=1.25cm,bottom=1.25cm,columnsep=1cm}

% The paracol package lets you typeset columns of text in parallel
\usepackage{paracol}

% Allow lualatex to load font files directly
% Ref: https://www.overleaf.com/learn/latex/Questions/I_have_a_custom_font_I%27d_like_to_load_to_my_document._How_can_I_do_this%3F
\usepackage{fontspec}

% Change the font if you want to, depending on whether
% you're using pdflatex or xelatex/lualatex
\ifxetexorluatex
  % If using xelatex or lualatex:
  \setmainfont{[RobotoSlab-Regular.ttf]}
  \setsansfont{[Lato-Regular.ttf]}
  \renewcommand{\familydefault}{\sfdefault}
\else
  % If using pdflatex:
  \usepackage[rm]{roboto}
  \usepackage[defaultsans]{lato}
  % \usepackage{sourcesanspro}
  \renewcommand{\familydefault}{\sfdefault}
\fi

% Change the colours if you want to
\definecolor{SlateGrey}{HTML}{2E2E2E}
\definecolor{LightGrey}{HTML}{666666}
\definecolor{DarkPastelRed}{HTML}{450808}
\definecolor{PastelRed}{HTML}{8F0D0D}
\definecolor{GoldenEarth}{HTML}{E7D192}
\colorlet{name}{black}
\colorlet{tagline}{PastelRed}
\colorlet{heading}{DarkPastelRed}
\colorlet{headingrule}{GoldenEarth}
\colorlet{subheading}{PastelRed}
\colorlet{accent}{PastelRed}
\colorlet{emphasis}{SlateGrey}
\colorlet{body}{LightGrey}

% Change some fonts, if necessary
\renewcommand{\namefont}{\Huge\rmfamily\bfseries}
\renewcommand{\personalinfofont}{\footnotesize}
\renewcommand{\cvsectionfont}{\LARGE\rmfamily\bfseries}
\renewcommand{\cvsubsectionfont}{\large\bfseries}


% Change the bullets for itemize and rating marker
% for \cvskill if you want to
\renewcommand{\itemmarker}{{\small\textbullet}}
\renewcommand{\ratingmarker}{\faCircle}

%% Use (and optionally edit if necessary) this .tex if you
%% want to use an author-year reference style like APA(6)
%% for your publication list
% \input{pubs-authoryear.cfg}

%% Use (and optionally edit if necessary) this .tex if you
%% want an originally numerical reference style like IEEE
%% for your publication list
\usepackage[backend=biber,style=ieee,sorting=ydnt]{biblatex}
%% For removing numbering entirely when using a numeric style
\setlength{\bibhang}{1.25em}
\DeclareFieldFormat{labelnumberwidth}{\makebox[\bibhang][l]{\itemmarker}}
\setlength{\biblabelsep}{0pt}
\defbibheading{pubtype}{\cvsubsection{#1}}
\renewcommand{\bibsetup}{\vspace*{-\baselineskip}}
\AtEveryBibitem{%
  \iffieldundef{doi}{}{\clearfield{url}}%
}


%% sample.bib contains your publications
\addbibresource{sample.bib}

\begin{document}
\name{Nham Hoai Le}
\tagline{Senior Platform Engineer}
%% You can add multiple photos on the left or right
%\photoR{2.8cm}{Globe_High}
% \photoL{2.5cm}{Yacht_High,Suitcase_High}

\personalinfo{%
  % Not all of these are required!
  \email{lehoainham@email.com}
  %\phone{000-00-0000}
  %\mailaddress{Åddrésş, Street, 00000 Cóuntry}
  \location{Ho Chi Minh, Vietnam}
  \homepage{www.nhamlh.space}
  \twitter{@nhamlh}
  \linkedin{nhamlh}
  \github{nhamlh}
  %\orcid{0000-0000-0000-0000}
  %% You can add your own arbitrary detail with
  %% \printinfo{symbol}{detail}[optional hyperlink prefix]
  % \printinfo{\faPaw}{Hey ho!}[https://example.com/]
  %% Or you can declare your own field with
  %% \NewInfoFiled{fieldname}{symbol}[optional hyperlink prefix] and use it:
  % \NewInfoField{gitlab}{\faGitlab}[https://gitlab.com/]
  % \gitlab{your_id}
  %%
  %% For services and platforms like Mastodon where there isn't a
  %% straightforward relation between the user ID/nickname and the hyperlink,
  %% you can use \printinfo directly e.g.
  % \printinfo{\faMastodon}{@username@instace}[https://instance.url/@username]
  %% But if you absolutely want to create new dedicated info fields for
  %% such platforms, then use \NewInfoField* with a star:
  % \NewInfoField*{mastodon}{\faMastodon}
  %% then you can use \mastodon, with TWO arguments where the 2nd argument is
  %% the full hyperlink.
  % \mastodon{@username@instance}{https://instance.url/@username}
}

\makecvheader
%% Depending on your tastes, you may want to make fonts of itemize environments slightly smaller
% \AtBeginEnvironment{itemize}{\small}


\begin{quote}
``Site Reliability Engineer with more than 8+ years of experience building and delivering internal service platform for organizations.''
\end{quote}

%% Set the left/right column width ratio to 7:3.
\columnratio{0.7}

% Start a 2-column paracol. Both the left and right columns will automatically
% break across pages if things get too long.
\begin{paracol}{2}
\cvsection{Experience}

\cvevent{Senior Infrastructure Engineer}{Y42}{Mar 2022 -- Ongoing}{Remote}
Build change management system to govern all infra components
Build observability platform to help quickly find and fix production issues using OpenTelemetry and Datadog.
Help service owners to ship their code confidently using Github Actions, Skaffold and Kustomize. Service owner can ship to production anytime they want.

\begin{itemize}
\item[--] Manage infrastructure built on Google Kubernetes Engine.
\item[--] Build CI/CD pipelines using Github Actions, Skaffold, Kustomize.
\item[--] Build observability platform using OpenTelemetry, Datadog.
\item[--] Build Infrastructure as Code with Terraform, Terragrunt, Atlantis.
\end{itemize}

\divider

\cvevent{Senior Platform Engineer}{Employment Hero}{Oct 2018 -- Feb 2022}{Remote}
Build unified compute platform to support all microservices.
\begin{itemize}
\item[--] Build Kubernetes based infrastructure on AWS for microservices.
\item[--] Build CI/CD pipelines using CircleCI, Flux, Helm.
\item[--] Build observability platform using Prometheus, Opsgenie, LogDNA, and Jaeger tracing.
\item[--] Build in-house tools to manage, automate and optimize infrastructure using shell script, Ruby and Golang.
\item[--] Platform governance: API gateway, service mesh, Postgresql, Redis, Kafka, etc.
\end{itemize}

\divider

\cvevent{DevOps Engineer}{Fossil Group}{Aug 2016 -- Sep 2018}{Vietnam}
\begin{itemize}
\item[--] Kubernetes the hard way on AWS using kops.
\item[--] Plan and build AWS based infrastructure for backend team.
\item[--] Plan and migrate systems running on AWS ECS to Kubernetes without downtime.
\item[--] Build in-house tools to govern, automate and optimize infrastructure using shell script, Python.
\item[--] Plan, design and create CI/CD process for development teams using Jenkins, Bitbucket pipelines.
\end{itemize}

\divider

\cvevent{Cloud Operation Engineer}{Posiba}{Jan 2016 - Jul 2016}{Vietnam}
\begin{itemize}
\item[--] Manage linux instances on AWS using Puppet.
\item[--] Monitor infrastructure components using Zabbix.
\item[--] Collect system and application logs using Fluentd, Elasticsearch and Kibana.
\item[--] Build in-house devops tools using shell script and Python.
\item[--] Help developers to roll out projects using devops tools: Puppet, Jenkins, Docker.
\end{itemize}

\divider

\cvevent{Linux System Administrator}{FPT Telecom}{Jul 2014 -- Jan 2016}{Vietnam}
\begin{itemize}
\item[--] Build and monitor Linux systems and network devices using Icinga2 and Cacti.
\item[--] Manage Linux servers with many roles: DNS clusters, email gateway, monitoring servers, etc.
\item[--] Develop Ansible playbooks to manage multiple linux servers.
\item[--] Manage internal users connectivity via Squid proxy
\item[--] Build, manage and backup VMware ESXi.
\end{itemize}

% use ONLY \newpage if you want to force a page break for
% ONLY the current column
%\newpage

%% Switch to the right column. This will now automatically move to the second
%% page if the content is too long.
\switchcolumn

\cvsection{Technologies used}

\cvtag{Golang}
\cvtag{Ruby}
\cvtag{Python}
\cvtag{Shell script}

\divider\smallskip

\cvtag{AWS}
\cvtag{GCP}

\divider\smallskip

\cvtag{Kubernetes}
\cvtag{Docker}
\cvtag{Helm}
\cvtag{Kustomize}
\cvtag{Flux}
\cvtag{Prometheus}

\divider\smallskip

\cvtag{gRPC}
\cvtag{Postgres}
\cvtag{Kafka}
\cvtag{Hydra}
\cvtag{Envoy}
\cvtag{OpenTelemetry}
\cvtag{Datadog}

\cvsection{Languages}

\cvskill{Vietnamese}{5}
\divider

\cvskill{English}{3}

%% Yeah I didn't spend too much time making all the
%% spacing consistent... sorry. Use \smallskip, \medskip,
%% \bigskip, \vspace etc to make adjustments.
\medskip

\cvsection{Education}

\cvevent{B.Sc.\ in Computer Network}{Ho Chi Minh University of Sciences}{2010 -- 2014}{}

% \divider

%\cvsection{Referees}

% \cvref{name}{email}{mailing address}
%\cvref{Name}{Institute}{a.beta@university.edu}

\end{paracol}


\end{document}
